\chapter{Conclusion}
The project of porting the Fedlang framework from Python to Go has proven to be a valuable endeavor, not only in achieving the primary objective of demonstrating Go's viability for federated learning systems but also in expanding the framework's capabilities. The transition to Go has resulted in a system that is more efficient, scalable, and easier to maintain, thanks to Go's concurrency model, efficient memory management, and faster serialization protocols.

Throughout the development process, the challenges of finding suitable libraries and ensuring seamless communication between the Go components and the existing Erlang middleware were successfully addressed. The use of the Ergo framework and the development of hybrid systems allowed for a smooth transition and ensured that the core functionalities of the original system were preserved and enhanced.

The introduction of a ring topology for distributed aggregation represents a significant improvement over the previous architecture. It not only reduces the communication bottleneck that can occur in traditional client-server setups but also opens up new possibilities for distributed algorithm developers to implement custom topologies that better suit their needs.

In conclusion, this project has laid a strong foundation for the future development of federated learning systems in Go. It has demonstrated that Go is not only a suitable alternative to Python but can also offer distinct advantages in terms of performance and scalability. The successful porting of the Fedlang framework serves as a valuable reference for future projects aiming to leverage the strengths of Go in distributed systems and middleware technologies.


